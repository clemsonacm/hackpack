\section{Segment Tree}
\index{segment tree}
\index{interval}

It is often useful to be aware of the range properties of a given array, such as: prefix sums, suffix sums, range minimum queries, range maximum queries, etc.
For simpler problems, a secondary array may suffice to calculate a property like the prefix sum of an array.
The prefix sum calculation only takes $O(n)$ time to perform, and any query thereafter takes only $O(1)$ time.
But any updates to the source dataset will require $O(n)$ time to update the prefix sum calculations.
Frequent changes will quickly show the pitfall of this approach.
A segment tree is a data structure for storing information about intervals that can be constructed in $O(n \log(n))$ time.
Whenever the source data changes, update times stay low at $O(\log n)$ time.

A segment tree is represented as an array of a size that is dependent on the dataset it is sourced from.
For a given array of size $n$, a segment tree constructed from it will use up to $2^{\lceil \log_2 (n)\rceil + 1} - 1$ space.
Once built, the structure of a segment tree cannot changed.

Generally, a segment tree supports two operations: update and query.
The update operation will account for changes to values in the source dataset.
The query operation allows for fast retrieval about segment information.
In the case of the prefix sum, the update operation will modify the tree's nodes such that it will contain correct information after changes, and the the query operation can return a sum for values in an arbitrary range.
