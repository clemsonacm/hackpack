\section{Set}
Sets are a data structure that are useful for determining if an element has been seen before or not.  Sets are Associative, Ordered, Set, Unique Keyed, and Allocator Aware\cite{cplusplus}.  Associative means that elements are referenced by some key and not by an absolute position in the data structure. Ordered means that internally the set follows a strict order based off the key.  Unique key means that there cannot be two keys with the same value.  Set means that the key is also the value. Allocator-aware means that the container knows who to dynamically allocate memory for itself.  Sets are generally implemented as a binary search tree.  See "unordered set" for a version of the set that is Unordered, but is based on hash tables.

\subsection{Reference}
\lstinputlisting[language=c++]{./set/set.cpp}

\subsection{Applications}
\begin{itemize}
    \item   Determining how many and what items are in one set are also in another
    \item   Determining how many and what items are in one set but \emph{not} in another
    \item   Filtering out non-unique inputs
\end{itemize}

\subsection{Contest Problem}

