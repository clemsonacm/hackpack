\usepackage{makeidx}
\usepackage{multicol}
\usepackage{geometry}
\usepackage{tabularx}
\geometry{margin=1in}

% Make links, footnotes, etc. clickable, and generate pdf bookmarks.
\usepackage{hyperref}
\hypersetup{
  pdfauthor={Clemson ACM, acm@cs.clemson.edu},
  pdfcreator={Clemson ACM, acm@cs.clemson.edu},
  pdftitle={Clemson ACM Hack Pack},
  pdfsubject={Clemson ACM Hack Pack},
  unicode=true,
  colorlinks=false,
  pdfborder=0 0 0,
  bookmarks=true
}

% Paragraph spacing should be easy!
\usepackage{parskip}

% Chapter title format: "X. Chaptertitle", not "Chapter X\nChaptertitle".
% Also, make \sections stand out a bit more.
\usepackage{titlesec}
\titleformat{\chapter}[block]{\normalfont\huge\bfseries}{\thechapter.}{1em}{\Huge}
\titleformat{\section}[block]{\normalfont\large\bfseries}{\thesection.}{1em}{\LARGE}

% Adjust spacing around headers to reduce empty space
\titlespacing*{\chapter}{0pt}{-22pt}{-1pt}
\titlespacing*{\section}{0pt}{10pt}{0pt}
\titlespacing*{\subsection}{0pt}{10pt}{0pt}
\titlespacing*{\subsubsection}{0pt}{0pt}{0pt}

% Citations should use superscript. It's nice.
\usepackage[superscript, biblabel]{cite}

% Use the Bera font family (based on Bitstream Vera) for text (bera), captions
% (berasans), and code (beramono).
\usepackage{bera}
\usepackage[scaled]{berasans}
\usepackage[scaled]{beramono}
\usepackage[T1]{fontenc}

% Code sections use the listings environment, supplemented by a caption (below)
\usepackage{color}
\usepackage{xcolor}
\usepackage{soul}
\usepackage{listings}
\lstset{
  basicstyle=\footnotesize\ttfamily, % Use a truetype, footnote-sized font.
  numbers=left,                      % line numbers go on the left
  numbersep=10pt,                    % how far the line-numbers are from the code
  tabsize=2,                         % sets default tabsize to 2 spaces
  extendedchars=true,                % lets you use non-ASCII characters; for 8-bits encodings only
  breaklines=true,                   % sets automatic line breaking
  keywordstyle=\bfseries,            % Keyword styling
  numberstyle=\ttfamily\color{gray}, % the style that is used for the line-numbers
  stringstyle=\color{gray},          % string literal style
  commentstyle=\itshape\color{gray}, % comment literal style
  title=\lstname,                    % show the filename of files included with \lstinputlisting
  frame=b,                           % show a frame along the bottom
  language=C++,                      % the default language of the code
  showspaces=false,                  % show spaces as space, not a litle underscore.
  showstringspaces=false             % same as above, in strings.
  showtabs=false,                    % same as above, with tabs.
  abovecaptionskip=5pt,
  belowcaptionskip=5pt,
  xleftmargin=17pt,
  framexleftmargin=17pt,
  framexrightmargin=5pt,
  framexbottommargin=4pt
}

% Listings caption configuration.
\usepackage{caption}
\DeclareCaptionFont{white}{\color{white}}
\DeclareCaptionFormat{listing}{\colorbox[cmyk]{0.43, 0.35, 0.35,0.01}{\parbox{\textwidth}{\vspace{1.5pt}\hspace{10pt}#1#2#3}}}
\captionsetup[lstlisting]{format=listing,labelfont=white,textfont=white, singlelinecheck=false, margin=0pt, font={bf, sf}}

% Turn on the powerful indexing features
\makeindex
