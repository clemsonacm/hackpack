\section{Dijkstra's Algorithm}\index{Shortest Path!Dijkstra}
Dijkstra’s Algorithm solves the single-source shortest path problem of finding the shortest paths between the source node and all other nodes in a connected graph with non-negative edge path costs.  
The graph can be both directed and undirected.  
It is commonly used to find the shortest path between a source and destination node.  
For graphs with negative weights, see Bellman-Ford Algorithm.
It is commonly inplemented using a priority queue.

\subsection{Applications}

\begin{itemize}
	\item  Finding the shortest paths between a source node and all other nodes in a graph
\end{itemize}

\subsection{Example Contest Problem: Milk Routing}
Farmer John's farm has an outdated network of M pipes (1 <= M <= 500) for pumping milk from the barn to his milk storage tank.  
He wants to remove and update most of these over the next year, but he wants to leave exactly one path worth of pipes intact, 
so that he can still pump milk from the barn to the storage tank.

The pipe network is described by N junction points (1 <= N <= 500), each of which can serve as the endpoint of a set of pipes.  
Junction point 1 is the barn, and junction point N is the storage tank.  
Each of the M bi-directional pipes runs between a pair of junction points, and has an associated latency 
(the amount of time it takes milk to reach one end of the pipe from the other) and capacity (the amount of milk per unit time
that can be pumped through the pipe in steady state).  
Multiple pipes can connect between the same pair of junction points.

For a path of pipes connecting from the barn to the tank, the latency of the path is the sum of the latencies of the pipes along 
the path, and the capacity of the path is the minimum of the capacities of the pipes along the path (since this is the "bottleneck" 
constraining the overall rate at which milk can be pumped through the path).  
If FJ wants to send a total of X units of milk through a path of pipes with latency L and capacity C, the time this takes is therefore L + X/C.

Given the structure of FJ's pipe network, please help him select a single path from the barn to the storage tank that will allow him to pump X units
of milk in a minimum amount of total time.

\subsubsection{Input Format}
\begin{itemize}
	\item Line 1: Three space-seperated integers: N M X (1 <= X <= 1,000,000).
	\item Line 2..1+M: Each line describes a pipe using 4 integers: I J L C.
			I and J (1 <= I,J <= N) are the juntion points at both ends of the pipe.
			L and C (1 <= L,C <= 1,000,000) give the latency and capacity of the pipe.
\end{itemize}

\subsubsection{Sample Input}
\acmlisting[caption=Milk Routing Input, label=Milk Routing Input]{./algorithms/dijkstra/problems/milk-routing/milkrouting.in}

\subsubsection{Output Format}
\begin{itemize}
	\item Line 1: The minimum amount of time it will take FJ to send milk along a single path, 
			rounded down to the nearest integer.
\end{itemize}

\subsubsection{Sample Output}
\acmlistingp[caption=Milk Routing Output, label=Milk Routing Output]{./algorithms/dijkstra/problems/milk-routing/milkrouting.out}

\subsubsection{Example Solution}
\acmlisting[caption=Milk Routing Solution, label=Milk Routing Solution]{./algorithms/dijkstra/problems/milk-routing/milkrouting.cpp}

\subsection{Example Contest Problem: Farm Tour}
When FJ's friends visit him on the farm, he likes to show them around. 
His farm comprises N (1 <= N <= 1000) fields numbered 1..N, the first of which contains his house and the Nth of which contains the big barn. 
A total M (1 <= M <= 10000) paths that connect the fields in various ways. 
Each path connects two different fields and has a nonzero length smaller than 35,000. 

To show off his farm in the best way, he walks a tour that starts at his house, potentially travels through some fields, and ends at the barn. 
Later, he returns (potentially through some fields) back to his house again. 

He wants his tour to be as short as possible, however he doesn't want to walk on any given path more than once. 
Calculate the shortest tour possible. 
FJ is sure that some tour exists for any given farm.

\subsubsection{Input Format}
\begin{itemize}
	\item Line 1: Two space-separated integers: N and M. 
	\item Lines 2..M+1: Three space-separated integers that define a path: The starting field, the end field, and the path's length. 
\end{itemize}

\subsubsection{Sample Input}
\acmlisting[caption=Farm Tour Input, label=Farm Tour Input]{./algorithms/dijkstra/problems/farm-tour/farmtour.in}

\subsubsection{Output Format}
\begin{itemize}
	\item Line 1: A single line containing the length of the shortest tour. 
\end{itemize}

\subsubsection{Sample Output}
\acmlistingp[caption=Milk Routing Output, label=Milk Routing Output]{./algorithms/dijkstra/problems/farm-tour/farmtour.out}












