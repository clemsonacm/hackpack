\section{Bloom Filter}
\index{Bloom filter}

A Bloom filter is a probabalistic data structure that, like a set, can keep track of elements that have been seen before.
It has the added advantage of not needing to retain the data after it has been added to the filter, thus, it is more conscientous of memory usage.
It works by utilizing a bit array of $m$ bits and $k$ hash functions.
The filter can return false positives, but will \textit{never} return false negatives.
Also, with the filter alone, once a member has been added to the set, it is not possible to remove it.
This can be worked around by implementing some form of counting.

To add data to the filter, simply feed it to the $k$ hash functions, mapping it to $k$ indices in the bit array and set the bits at the $k$ indices to 1.
To test whether some data has passed through the filter, feed it to the hash functions to compute $k$ indices.
Check the values at these positions.
If a 0 is encountered, it can be immediately determined that the specified data did not pass through the filter.
If, after checking all $k$ positions in the bit array, no 0s are encountered, then the data \textbf{possibly} passed through the filter.

One can never posit with 100\% accuracy that the data is a member of the set by testing the filter alone.
Because of that fact, some additional checking is necessary to provide a definite answer to a query.
